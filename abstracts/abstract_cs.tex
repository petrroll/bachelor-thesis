%% Settings for single-side (simplex) printing
% Margins: left 40mm, right 25mm, top and bottom 25mm
% (but beware, LaTeX adds 1in implicitly)
\documentclass[12pt,a4paper]{report}
\setlength\textwidth{145mm}
\setlength\textheight{247mm}
\setlength\oddsidemargin{15mm}
\setlength\evensidemargin{15mm}
\setlength\topmargin{0mm}
\setlength\headsep{0mm}
\setlength\headheight{0mm}
% \openright makes the following text appear on a right-hand page
\let\openright=\clearpage



%% Generate PDF/A-2u
\usepackage[a-2u]{pdfx}
\usepackage[utf8]{inputenc}

%% Prefer Latin Modern fonts
\usepackage{lmodern}

\begin{document}
Cílem této práce je navrhnout a implementovat podporu pro generátory v rámci Peachpie projektu, kompilátoru z PHP do CIL. Generátory jsou nejjednodušší formou metod, které při opětovném zavolání pokračují ze stejného stavu, ve kterém dříve skončily. 

Referenční interpretr jazyka PHP Zend Engine podporuje generátory nativně. Nejen kvůli tomu jsou generátory v PHP poměrně silné a umožňují například přerušení vykonávání prakticky na jakémkoliv místě. CIL naopak přímou podporu pro generátory nemá, jazyky nad ním postavené (např. C\# či F\#) je tedy musí implementovat kupříkladu přepisem generátorových metod na stavové automaty. 

V práci rozebereme návrh a následnou implementaci generátorů pomocí transformace sémantického stromu a konkrétní CIL reprezentaci případných nových sémantických objektů s cílem zachovat maximální možnou kompatibilitu se sémantikou PHP generátorů. Také je srovnáme s generátory v jazyce C\#, jehož hlavní implementace je taktéž postavena nad CIL.

\end{document}
