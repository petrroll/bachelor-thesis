%%% The main file. It contains definitions of basic parameters and includes all other parts.

%% Settings for single-side (simplex) printing
% Margins: left 40mm, right 25mm, top and bottom 25mm
% (but beware, LaTeX adds 1in implicitly)
\documentclass[12pt,a4paper]{report}
\setlength\textwidth{145mm}
\setlength\textheight{247mm}
\setlength\oddsidemargin{15mm}
\setlength\evensidemargin{15mm}
\setlength\topmargin{0mm}
\setlength\headsep{0mm}
\setlength\headheight{0mm}
% \openright makes the following text appear on a right-hand page
\let\openright=\clearpage

%% Settings for two-sided (duplex) printing
% \documentclass[12pt,a4paper,twoside,openright]{report}
% \setlength\textwidth{145mm}
% \setlength\textheight{247mm}
% \setlength\oddsidemargin{14.2mm}
% \setlength\evensidemargin{0mm}
% \setlength\topmargin{0mm}
% \setlength\headsep{0mm}
% \setlength\headheight{0mm}
% \let\openright=\cleardoublepage

%% Generate PDF/A-2u
\usepackage[a-2u]{pdfx}

%% Character encoding: usually latin2, cp1250 or utf8:
\usepackage[utf8]{inputenc}

%% Prefer Latin Modern fonts
\usepackage{lmodern}

%% Further useful packages (included in most LaTeX distributions)
\usepackage{amsmath}        % extensions for typesetting of math
\usepackage{amsfonts}       % math fonts
\usepackage{amsthm}         % theorems, definitions, etc.
\usepackage{bbding}         % various symbols (squares, asterisks, scissors, ...)
\usepackage{bm}             % boldface symbols (\bm)
\usepackage{graphicx}       % embedding of pictures
\usepackage{fancyvrb}       % improved verbatim environment
\usepackage{natbib}         % citation style AUTHOR (YEAR), or AUTHOR [NUMBER]
\usepackage[nottoc]{tocbibind} % makes sure that bibliography and the lists
			    % of figures/tables are included in the table
			    % of contents
\usepackage{dcolumn}        % improved alignment of table columns
\usepackage{booktabs}       % improved horizontal lines in tables
\usepackage{paralist}       % improved enumerate and itemize
\usepackage[usenames]{xcolor}  % typesetting in color
\usepackage{minted}
\usepackage{hyperref}

%%% Basic information on the thesis

% Thesis title in English (exactly as in the formal assignment)
\def\ThesisTitle{Compilation of a dynamic language Generators into MSIL}

% Author of the thesis
\def\ThesisAuthor{Petr Houška}

% Year when the thesis is submitted
\def\YearSubmitted{2017}

% Name of the department or institute, where the work was officially assigned
% (according to the Organizational Structure of MFF UK in English,
% or a full name of a department outside MFF)
\def\Department{Department of Software Engineering}

% Is it a department (katedra), or an institute (ústav)?
\def\DeptType{Department}

% Thesis supervisor: name, surname and titles
\def\Supervisor{Mgr. Jakub Míšek}

% Supervisor's department (again according to Organizational structure of MFF)
\def\SupervisorsDepartment{Department of Software Engineering}

% Study programme and specialization
\def\StudyProgramme{Computer Science}
\def\StudyBranch{General Computer Science}

% An optional dedication: you can thank whomever you wish (your supervisor,
% consultant, a person who lent the software, etc.)
\def\Dedication{%
I would like to thank my advisor Mgr. Jakub Míšek for his valuable advice and guidance he has given me for this thesis. I would also like to thank him for starting the Peachpie project in the first place and bringing me on despite the disadvantages mentoring a student who is writing a bachelor thesis inherently brings. 

This thesis would also not be possible without the endless support of my parents, friends, and classmates and the endless encouragement of my girlfriend. All of them and practically everyone else who I had the pleasure to meet during my studies deserve an acknowledgment.
}

% Abstract (recommended length around 80-200 words; this is not a copy of your thesis assignment!)
\def\Abstract{%
The goal of this thesis is to design and implement support for generators within the Peachpie framework, a PHP to CIL compiler. Generators are the simplest form of methods that resume from the same state in which they returned earlier when called repeatedly. The reference PHP interpreter Zend engine supports generators natively. Due to that fact generators in PHP are support a number of features not usually seen in other languages. CIL on the other hand does not have a native support for generators. Therefore, languages build on top of CIL (e.g. C\#, F\#) have to implement them by other means such as by rewriting the original generator methods into state machines. In this thesis we will design and implement support for generators through semantic tree transformations. All that with the intention of keeping the maximum possible compatibility with reference PHP generators. We will also make a comparison to generators in C\# whose main implementation also uses CIL as a backend.
}

% 3 to 5 keywords (recommended), each enclosed in curly braces
\def\Keywords{%
	{compiler} {php} {msil} {.net} {generators} {roslyn} {peachpie}
}


%% The hyperref package for clickable links in PDF and also for storing
%% metadata to PDF (including the table of contents).
%% Most settings are pre-set by the pdfx package.
\hypersetup{unicode}
\hypersetup{breaklinks=true}

% Definitions of macros (see description inside)
\include{macros}

% Title page and various mandatory informational pages
\begin{document}
\include{title}

%%% A page with automatically generated table of contents of the bachelor thesis

\tableofcontents

%%% Each chapter is kept in a separate file
\chapwithtoc{Introduction}

Despite a slight  decline in recent years \citep{Tiobe}, PHP is still one of the main languages used for server side programming on the web \citep{Stack}. Its only two relevant implementations are the reference and almost exclusively used Zend engine\footnote{\href{http://www.zend.com/en/community/php}{zend.com/en/community/php}} and the slowly emerging HHVM by Facebook\footnote{\href{http://hhvm.com/}{hhvm.com/}}. Both of them are standalone virtual machines and neither of them supports easy interfacing with the outside world. Hence, it is quite difficult to share code between a web backend and, for example, a~mobile or traditional desktop application.

Fortunately, there is a solution in the form of the Peachpie project\footnote{\href{http://www.peachpie.io/}{peachpie.io/}} that is being researched at the Charles University. The project aims to provide a compiler from PHP to “.NET bytecode” CIL\footnote{Chapter \ref{CIL}} and a reimplementation of the PHP base class library, thus creating a bridge between PHP and the whole .NET ecosystem. Due to the fact that it is a full compiler that takes PHP sources and outputs .NET assemblies indistinguishable from those created by other .NET languages compilers (e.g C\#, F\# or IronPython), it provides a both-way interoperability. It enables both calling normal unmodified .NET libraries from PHP and vice versa. Also, thanks to an extensive compile-time type analysis and the proven .NET just in time compiler (RiuJIT) it achieves better performance than the reference Zend engine in certain operations \citep{PchpBenchBlog, PchpBenchSite}.

PHP, like many other modern languages, has a first class support for generators. In short, generators are methods that, when called repeatedly, resume the computation from the very place and with the same state they returned at previously. They are usually used for generating large sequences of data lazily, hence the name generators. Since the execution state gets saved automatically on the special pause and return places (usually called yields), one can write an algorithm as if the sequence were being created at once and only insert yields at appropriate times, e.g. when a new item gets created. The language handles the rest. Each subsequent call to the generator method resumes the computation from the last evaluated yield and continues to the next one, e.g. creating a new element each time. 

The Zend engine has a native support for generators. It intrinsically understands yields and is, on their evaluation, able to save the state of current execution \citep{ZendGen}. CIL has no such first class support. For that reason, languages built on top of the CIL have to implement generators through other means \citep{CSharpGen} - usually by rewriting generator methods into state machines with the explicit state saving before each yield and a state retrieval in the beginning.

This thesis describes the design and implementation of a support for PHP generators within the Peachpie compiler through semantic tree transformations, an implementation of new semantic tree nodes, and extensions to the Peachpie runtime library. In the implementation parts, the thesis not only tries to plainly cover the code, but also to depict the decision process that led to choosing certain approaches over others. Throughout the work, we will compare our approach with the one taken by the C\# team and its compiler Roslyn. C\# was chosen as a reference language due to  being the prominent language in .NET platform.

While the goal is to implement a support for generators with as much original PHP semantic as possible, due to the scope of this work we will not discuss the specific implementation of all PHP generator features. Namely, we will not cover handling yields in exception control blocks (try, catch, finally) in detail and will leave its implementation for future work.

\secwithtoc{Thesis structure}
This thesis is divided into seven chapters. The first one covers general concepts of generators both in PHP and in other languages, explaining what they are, what features and limitations they have, and where they stand with regards to iterators. 

The second chapter briefly introduces the .NET platform and its intermediate language CIL. The third is entirely about the Peachpie project. It describes its architecture, focusing mainly on the semantic tree data structure and CIL emit phase of the compiler. In the fourth chapter, we examine how generators are implemented in C\#’s Roslyn and PHP’s Zend engine. Roslyn’s approach is particularly important, because it serves as a basis for our own implementation.
 
Generators within Peachpie is the focus of the fifth chapter, which itself is further divided into five subsections. The first part describes an implementation of generators limited to circa C\# generators. It builds on the theoretical basis described in the previous section about Roslyn’s approach. The second one proposes a theoretical algorithm to handle yield as an expression. The third subsection discusses the implementation of said algorithm within Peachpie. In the fourth part, we briefly mention the possible solutions for yields in exception handling blocks. Finally, the fifth subsection is about possible future work that could be done for generator support within Peachpie. 

The sixth chapter concludes and summarizes the whole thesis. Ultimately, the final chapter provides a lightweight user documentation for the Peachpie project and an overview of attachments.


\chapter{Generators}

\section{Iterators}

\section{Generators universally}

\section{Generators in other languages}




\chapter{.NET platform}

The .NET platform, or any platform implementing the open CLI standard\footnote{\citep{CLIEcma}}, stands on four pillars (\autoref{fig2.1:CLI}). The low level intermediate language CIL, the higher level languages such as C\# and F\# and their compilers to CIL, the base class library known as .NET framework, and - last but not least - the common language runtime, CLR, that actually executes the intermediate code.

\begin{figure}[h]
	\centering	
	\includegraphics[scale=0.75]{../img/2_1_CLI}	
	\caption{Common language infrastructure.}
	\label{fig2.1:CLI}
\end{figure}

We will talk about the C\# and Visual Basic compiler Roslyn later, but neither the base class library nor the CLR will be covered extensively in this thesis. The common intermediate language, however, will be discussed in detail. 

\section{Common intermediate language}\label{CIL}

CIL is an assembler defined by the common language infrastructure \citep{CLIEcma}\nocite{CSharpEcma} to be a shared basis for all CLI languages (C\#, F\#, IronPython, …) and runtime implementations (.NET, Mono, dotGNU, …). It is platform independent and as such does not natively run on any CPU architecture. Instead, it must be either translated to the target platform’s native code beforehand or - more commonly - executed by a virtual machine such as CLR.

Despite being an assembler, thus inherently low-level, CIL is actually object oriented and so has a deep understanding of reference types. Its instruction set reflects this with means to create new instances, access their members, and so on. The CLI specification also dictates that, by default, the CIL should be memory safe. 

\subsection{Evaluation stack}

CIL is a stack based assembler, therefore without the notion for registers. Instead, it defines a virtual evaluation stack. There are basically two types of instructions in CIL. Firstly, there are memory handling ones that either pop a value from the stack and store it in memory or load a value from memory and push it to the top. Secondly, there are instructions that actually do some processing. These pop a few values from the stack, process them in some way, and then store the result on the top of the stack.

There are a few important things to note about the evaluation stack \citep[Sec. I.12.4]{CLIEcma}. Firstly, all parameters and local variables actually live there. They are not ordinary stack values, though. Their place gets reserved and later cleaned automatically and they are not accessible through the normal push/pop instructions. Instead there are dedicated instructions to work with them.

Secondly, when exiting a function, the stack cannot contain anything but the returned value. Thirdly, there are instructions only to work with its top. There is no way to query all the elements in the stack, get its height, or to completely save or load it to/from memory.

Lastly, while not a rule, the stack is generally used as a store for temporal values instead of proper local variables. For example, an expression $2 + 3 * 5$ (\autoref{list2.1}) would usually result in the load of constants $2$, $3$, and $5$, a multiplication operation $(3 * 5)$ (see \emph{IL\_0009}), at which point the stack would contain $2$ and $15$, and finally a plus operation (see \emph{IL\_000a}) that would leave the stack with $17$ at its top.

\begin{listing}[h]
	\caption{Simple method in C\# and CIL.}
	\label{list2.1}
\begin{minted}{csharp}
public void M(int a) {
  int b = 3;     // IL_0001 & IL_0002
  int c = 5;     // IL_0003 & IL_0004   
  G(a + b * c);  // IL_0005 - IL_000b
}
public int G(int a){/*Something*/}
\end{minted}
\begin{minted}[breaklines=true]{js}
.method public hidebysig instance void M (int32 a) 
cil managed {
  .maxstack 4
  .locals init ([0] int32, [1] int32)
  IL_0000: nop       // Do nothing (No operation)
  IL_0001: ldc.i4.3  // Push 3 onto the stack as int32
  IL_0002: stloc.0   // Pop value from stack to local variable 0
  IL_0003: ldc.i4.5  // Push 5 onto the stack as int32
  IL_0004: stloc.1   // Pop value from stack to local variable 1
  IL_0005: ldarg.0   // Load argument 0 (this) onto the stack
  IL_0006: ldarg.1   // Load argument 1 onto the stack
  IL_0007: ldloc.0   // Load local variable 0 onto stack
  IL_0008: ldloc.1   // Load local variable 1 onto stack
  IL_0009: mul       // Multiply values
  IL_000a: add       // Add two values, returning a new value
  IL_000b: call instance int32 C::G(int32) // Call method indicated on the stack with arguments
  IL_0010: pop       // Pop value (returned by G) from the stack
  IL_0011: ret       // Return from method, possibly with a value
} // end of method C::M
\end{minted}
\end{listing}

All of these mean that you cannot simply pause and save the execution of a method at an arbitrary point with just one or even a few CIL instructions. To completely capture the current state, you not only need to save all the local variables and parameters somewhere off the stack, but you must also do the same for every temporal value that might at that moment live on the stack. And there is no simple way to query what is there. You either need to construct the information in some other way or restrain yourself to saving the state only when the stack is empty.

\subsection{Exception handling}

The last notable thing about CIL is that it has a notion of exceptions and their handling blocks. Try, catch, and finally are all first class citizens in the language and are bound by a number of rules \citep[Sec. I.12.4]{CLIEcma}.

CIL does not permit jumping / branching into any exception handling block\footnote{\citep[Sec. I.12.4.2.8.2.7]{CLIEcma}} unless the source of the jump / branch is within the same block. You can only enter catch and finally regions through the proper exception handling mechanism. And lastly, to leave any of them\footnote{\citep[Sec. I.12.4.2.8.2.8]{CLIEcma}}, you need to do it via a designed instruction that, in case of try and catch blocks, ensures any potential finally region gets run. Therefore, you can neither jump in the middle of a try block nor execute a catch / finally block without throwing a proper exception first.



\chapter*{Conclusion}
\addcontentsline{toc}{chapter}{Conclusion}

In previous four sections we have first described the fundamental concepts required for understanding this thesis, then designed an algorithm to support our feature, provided an overview of said algorithm's implementation, and, in the end, proposed possible expansions.

While the work on generators support within the Peachpie compiler is by no means done, the shipped implementation provides a good foundation that can stand on its own. It brings support for all generator’s features, except for yields in exception handling blocks. And while that is an useful feature, it is a more of an extension of generators than its fundamental building block. Other than that our implementation mimics the reference semantics faithfully, while expanding upon the featureset usual in other CLI based languages such as in C\#.

The goal of using as much existing architecture as possible and not creating unnecessary abstractions just for generators was also achieved. While there is still room for an improvement, all generators specific code is either cleanly separated or abstracted to be used by other compiler components as well. 
Lastly, while not an explicitly stated goal, the compilation of generators is efficient. It does not introduce any new semantic tree or syntax tree traversals and only slightly increases the memory required for the binding phase. Due to the separation of all specific logic to a special binder, it has absolutely no impact on binding, and thus compiling, non-generator methods.

In conclusion, this thesis and the attached implementation fulfill all goals set by both the thesis assignment and us in the introduction section. On top of that, it brings a self-contained functionality to a popular open source project.

%%% Attachments to the bachelor thesis, if any. Each attachment must be
%%% referred to at least once from the text of the thesis. Attachments
%%% are numbered.
%%%
%%% The printed version should preferably contain attachments, which can be
%%% read (additional tables and charts, supplementary text, examples of
%%% program output, etc.). The electronic version is more suited for attachments
%%% which will likely be used in an electronic form rather than read (program
%%% source code, data files, interactive charts, etc.). Electronic attachments
%%% should be uploaded to SIS and optionally also included in the thesis on a~CD/DVD.
%%% Allowed file formats are specified in provision of the rector no. 23/2016.
\chapwithtoc{Attachments}

Attached to this thesis is a snapshot of Peachpie project’s git repository. It contains not only the implementation that was done as the practical part of this thesis but also the rest of the complete project. A more up to date version can be found on github\footnotemark.

To query only commits done by the author of this thesis please filter out author \emph{Petr Houška} or email \emph{houskape@gmail.com}.

\footnotetext{
	\href{https://github.com/peachpiecompiler/peachpie}{github.com/peachpiecompiler/peachpie}
}

\secwithtoc{Compilation}
The project’s only implicit dependency is .NET Core runtime and optionally its CLI SDK. If you want to compile the project yourself you can download both of them from the official site\footnotemark, for Linux, Windows, or MacOSX.

After obtaining the .NET Core SDK please navigate to the folder with the Peachpie repository in your favourite terminal and:

\begin{minted}[breaklines=true]{text}
dotnet restore  //download all external packages required
dotnet build    //build the complete solution
\end{minted}

\footnotetext{
	\href{https://www.microsoft.com/net/download/core}{microsoft.com/net/download/core}
}

\secwithtoc{Structure}
There are three components relevant for this thesis within the repository. The compiler binaries, the compiler implementation, and the generators tests. Below are listed paths to them and in case of the compiler’s implementation also to some files containing the majority of our work to support generators.

\begin{enumerate}
	\item \label{peach}src/Compiler/peach	
	\item src/CodeAnalysis
	\begin{enumerate}
		\item ./Semantics/SemanticsBinder.cs
		\item ./Semantics/Graph/BuilderVisitor.cs
		\item src/Peachpie.Runtime/std/Generator.cs
	\end{enumerate}
	\item tests/generators
\end{enumerate}

\secwithtoc{Manual testing}
To compile an arbitrary PHP file into a .NET assembly with Peachpie invoke the compiler with a path to the PHP file as its first argument. The compiler assembly resides at aforementioned path and is called peach.exe or peach.dll depending of whether it was compiled for full .NET framework or .NET Core.

\begin{minted}[breaklines=true]{text}
$\src\Compiler\peach> dotnet run .\test.php
\end{minted}

Please do note that an assembly compiled this way will require PHP runtime libraries to run. These libraries can be found, for example, in the bin output of the \hyperref[peach]{compiler} (peach) project.

Alternatively it is possible to use a Peachpie console application sample\footnotemark. It includes a .msbuildproj file that configures the .NET Core CLI to download and use both the Peachpie compiler toolchain and required runtime libraries automatically.\footnotetext{
	\href{https://github.com/iolevel/peachpie-samples/tree/master/console-application}{github.com/iolevel/peachpie-samples/tree/master/console-application}
} 
More about that approach can be found on a peachpie blog\footnotemark.



\footnotetext{
	\href{http://www.peachpie.io/2017/04/tutorial-vs2017.html}{peachpie.io/2017/04/tutorial-vs2017.html}
}

\secwithtoc{Automatic testing}
The Peachpie project includes a comprehensive set of automatic tests. These consist of PHP files that get compiled by the Peachpie compiler and run by a .NET runtime. If there is a PHP runtime present in the current path environment variable they get run by it as well. The results are then compared to ensure Peachpie compilation keeps the original PHP semantics and is, in terms of runtime behaviour, indistinguishable from the reference implementation.

There is a number tests created as part of this thesis that ensure the implementation of generators support works correctly. They are located in a subfolder tests/generators. While they are in no particular order it is generally true that the higher their number the more complex aspect of generators they test. Below is a command that invokes all peachpie tests, including generator ones.

\begin{minted}[breaklines=true]{text}
$\src\Tests\Peachpie.ScriptTests> dotnet test
\end{minted}


Please do note that two tests usually fail on some machines because of encoding issues. 









%%% Bibliography
\include{bibliography}

%%% Figures used in the thesis (consider if this is needed)
\listoffigures

%%% Tables used in the thesis (consider if this is needed)
%%% In mathematical theses, it could be better to move the list of tables to the beginning of the thesis.
% \listoftables

%%% Abbreviations used in the thesis, if any, including their explanation
%%% In mathematical theses, it could be better to move the list of abbreviations to the beginning of the thesis.
\chapwithtoc{List of Abbreviations}

\begin{description}
	
	\item[CLI] Common language infrastructure, open standard for runtime environment implemented by .NET, Mono, and others.
	
	\item[CIL] Common intermediate language, object oriented assembler defined by \emph{CLI} (also known as \emph{MSIL} or \emph{IL}).
	
	\item[CLR] Common language runtime, virtual machine implementing the execution engine specified by \emph{CLI}.
	
	\item[DLR] Dynamic language runtime, set of libraries providing compiler and runtime services for dynamic languages build on top of \emph{CLR}.
	
	\item[AST] Abstract syntax tree, structured representation of the source code.
	
	\item[CFG] Control flow graph, a semantic graph representing a method.
	
\end{description}




\openright
\end{document}
