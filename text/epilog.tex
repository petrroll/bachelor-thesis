\chapter*{Conclusion}
\addcontentsline{toc}{chapter}{Conclusion}

In previous four sections we have first described the fundamental concepts required for this thesis to even make sense, then designed an algorithm to support our feature, provided an overview of said algorithm's implementation, and, in the end, proposed possible expansions.

While the work on generators support within the Peachpie compiler is by no means done the shipped implementation provides a good foundation that can stand on its own. It brings support for all generator’s features except for yields in exception handling blocks. And while that is an useful feature it is a more of an extension of generators than its fundamental building block. Other than that our implementation mimics the reference semantics faithfully while expanding upon the featureset usual in other CLI based languages such as in C\#.

The goal of using as much existing architecture as possible and not creating unnecessary abstractions just for generators was also achieved. While there is still room for improvement all generators specific code is either cleanly separated or repurposed for use by other compiler components.

Lastly, while not an explicitly stated goal the compilation of generators is efficient. It does not introduce any new semantic tree or syntax tree traversals and only slightly increases the memory required for the binding phase. Due to separating all specific logic to a special binder it has absolutely no impact on binding and thus compiling non-generator methods.

In conclusion this thesis and the attached implementation fulfill all goals set by both the thesis assignment and us in the introduction section. On top of that it brings a self-contained functionality to a popular open source project.
